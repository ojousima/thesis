%%%%%%%%%%%%%%%%%%%%%%%%%%%%%%%%%%%%%%%%%%%%%%%%%%%%%%%%%%%%%%%%%%%%
%%%%%%%%%%%%%%%%%%%%%%%%%%%%%%%%%%%%%%%%%%%%%%%%%%%%%%%%%%%%%%%%%%%%
%%                                                                %%
%% An example for writting your thesis using LaTeX                %%
%% Original version by Luis Costa,  changes by Perttu Puska       %%
%% Support for Swedish added 15092014                             %%
%%                                                                %%
%% This example consists of the files                             %%
%%         thesistemplate.tex (versio 2.01)                       %%
%%         opinnaytepohja.tex (versio 2.01) (for text in Finnish) %%
%%         aaltothesis.cls (versio 2.01)                          %%
%%         kuva1.eps                                              %%
%%         kuva2.eps                                              %%
%%         kuva1.pdf                                              %%
%%         kuva2.pdf                                              %%
%%                                                                %%
%%                                                                %%
%% Typeset either with                                            %%
%% latex:                                                         %%
%%             $ latex opinnaytepohja                             %%
%%             $ latex opinnaytepohja                             %%
%%                                                                %%
%%   Result is the file opinnayte.dvi, which                      %%
%%   is converted to ps format as follows:                        %%
%%                                                                %%
%%             $ dvips opinnaytepohja -o                          %%
%%                                                                %%
%%   and then to pdf as follows:                                  %%
%%                                                                %%
%%             $ ps2pdf opinnaytepohja.ps                         %%
%%                                                                %%
%% Or                                                             %%
%% pdflatex:                                                      %%
%%             $ pdflatex opinnaytepohja                          %%
%%             $ pdflatex opinnaytepohja                          %%
%%                                                                %%
%%   Result is the file opinnaytepohja.pdf                        %%
%%                                                                %%
%% Explanatory comments in this example begin with                %%
%% the characters %%, and changes that the user can make          %%
%% with the character %                                           %%
%%                                                                %%
%%%%%%%%%%%%%%%%%%%%%%%%%%%%%%%%%%%%%%%%%%%%%%%%%%%%%%%%%%%%%%%%%%%%
%%%%%%%%%%%%%%%%%%%%%%%%%%%%%%%%%%%%%%%%%%%%%%%%%%%%%%%%%%%%%%%%%%%%

%% Uncomment one of these:
%% the 1st when using pdflatex, which directly typesets your document in
%% pdf (use jpg or pdf figures), or
%% the 2nd when producing a ps file (use eps figures, don't use ps figures!).
\documentclass[english,12pt,a4paper,pdftex,elec,utf8]{aaltothesis}
%\documentclass[english,12pt,a4paper,dvips]{aaltothesis}

%% To the \documentclass above
%% specify your school: arts, biz, chem, elec, eng, sci
%% specify the character encoding scheme used by your editor: utf8, latin1

%% Use one of these if you write in Finnish (see the Finnish template):
%%
%\documentclass[finnish,12pt,a4paper,pdftex,elec,utf8]{aaltothesis}
%\documentclass[finnish,12pt,a4paper,dvips]{aaltothesis}

\usepackage{graphicx}

%% Use this if you write hard core mathematics, these are usually needed
\usepackage{amsfonts,amssymb,amsbsy}

%% Use the macros in this package to change how the hyperref package below 
%% typesets its hypertext -- hyperlink colour, font, etc. See the package
%% documentation. It also defines the \url macro, so use the package when 
%% not using the hyperref package.
%%
%\usepackage{url}

%% Use this if you want to get links and nice output. Works well with pdflatex.
\usepackage{hyperref}
\hypersetup{pdfpagemode=UseNone, pdfstartview=FitH,
  colorlinks=true,urlcolor=red,linkcolor=blue,citecolor=black,
  pdftitle={Default Title, Modify},pdfauthor={Your Name},
  pdfkeywords={Modify keywords}}

%Use for degree symbol
\usepackage{gensymb}

%for color
\usepackage{color}

%% All that is printed on paper starts here
\begin{document}

%% Change the school field to specify your school if the automatically 
%% set name is wrong
% \university{aalto-yliopisto}
% \university{aalto University}
% \school{Sähkötekniikan korkeakoulu}
% \school{School of Electrical Engineering}

%% Only for B.Sc. thesis: Choose your degree programme. 
%\degreeprogram{Electronics and electrical engineering}
%%

%% ONLY FOR M.Sc. AND LICENTIATE THESIS: Specify your department,
%% professorship and professorship code. 
%%
\department{Department of Automation and Systems Technology}
\professorship{Smart products}
%%

%% Valitse yksi näistä kolmesta
%%
%% Choose one of these:
%\univdegree{BSc}
\univdegree{MSc}
%\univdegree{Lic}

%% Your own name (should be self explanatory...)
\author{Otso Jousimaa}

%% Your thesis title comes here and again before a possible abstract in
%% Finnish or Swedish . If the title is very long and latex does an
%% unsatisfactory job of breaking the lines, you will have to force a
%% linebreak with the \\ control character. 
%% Do not hyphenate titles.
%% 
\thesistitle{Energy Harvester Design for Intelligent Tire Systems}

\place{Espoo}

%% For B.Sc. thesis use the date when you present your thesis. 
%% 
%% Kandidaatintyön päivämäärä on sen esityspäivämäärä! 
\date{XX.XX.XXXX}

%% B.Sc. or M.Sc. thesis supervisor 
%% Note the "\" after the comma. This forces the following space to be 
%% a normal interword space, not the space that starts a new sentence. 
%% This is done because the fullstop isn't the end of the sentence that
%% should be followed by a slightly longer space but is to be followed
%% by a regular space.
%%
\supervisor{Prof.\ Arto Visala} %{Prof.\ Pirjo Professori}

%% B.Sc. or M.Sc. thesis advisors(s). You can give upto two advisors in
%% this template. Check with your supervisor how many official advisors
%% you can have.
%%
%\advisor{Prof.\ Pirjo Professori}
\advisor{D.Sc.\ (Tech.) Ari Tuononen}
\advisor{M.Sc.\ Yi Xiong}

%% Aalto logo: syntax:
%% \uselogo{aaltoRed|aaltoBlue|aaltoYellow|aaltoGray|aaltoGrayScale}{?|!|''}
%%
%% Logo language is set to be the same as the document language.
%% Logon kieli on sama kuin dokumentin kieli
%%
\uselogo{aaltoRed}{''}

%% Create the coverpage
%%
\makecoverpage


%% Note that when writting your master's thesis in English, place
%% the English abstract first followed by the possible Finnish abstract

%% English abstract.
%% All the information required in the abstract (your name, thesis title, etc.)
%% is used as specified above.
%% Specify keywords
%%
%% Kaikki tiivistelmässä tarvittava tieto (nimesi, työnnimi, jne.) käytetään
%% niin kuin se on yllä määritelty.
%% Avainsanat
%%
\keywords{For keywords choose concepts that are central to your thesis}
%% Abstract text
\begin{abstractpage}[english]
  Your abstract in English. Try to keep the abstract short; approximately 
  100 words should be enough. The abstract explains your research topic, 
  the methods you have used, and the results you obtained.  
  Your abstract in English. Try to keep the abstract short; approximately 
  100 words should be enough. The abstract explains your research topic, 
  the methods you have used, and the results you obtained.  

  Your abstract in English. Try to keep the abstract short; approximately 
  100 words should be enough. The abstract explains your research topic, 
  the methods you have used, and the results you obtained.  
  Your abstract in English. Try to keep the abstract short; approximately 
  100 words should be enough. The abstract explains your research topic, 
  the methods you have used, and the results you obtained.  
\end{abstractpage}

%% Force a new page so that the possible English abstract starts on a new page
%%
%% Pakotetaan uusi sivu varmuuden vuoksi, jotta 
%% mahdollinen suomenkielinen ja englanninkielinen tiivistelmä
%% eivät tule vahingossakaan samalle sivulle
\newpage
%
%% Abstract in Finnish.  Delete if you don't need it. 
\thesistitle{Energy Harvester Design for Intelligent Tire Systems}
\advisor{DI Yi Xiong}
\advisor{TkT Ari Tuononen}
\degreeprogram{Automaatio- ja systeemitekniikka}
\department{Automaatio- ja systeemitekniikan laitos}
\professorship{Älykkäät tuotteet}
%% Avainsanat
\keywords{Vastus, Resistanssi,\\ Lämpötila}
%% Tiivistelmän tekstiosa
\begin{abstractpage}[finnish]
  Tiivistelmässä on lyhyt selvitys (noin 100 sanaa)
  kirjoituksen tärkeimmästä sisällöstä: mitä ja miten on tutkittu,
  sekä mitä tuloksia on saatu. 
  Tiivistelmässä on lyhyt selvitys (noin 100 sanaa)
  kirjoituksen tärkeimmästä sisällöstä: mitä ja miten on tutkittu,
  sekä mitä tuloksia on saatu. 

  Tiivistelmässä on lyhyt selvitys (noin 100 sanaa)
  kirjoituksen tärkeimmästä sisällöstä: mitä ja miten on tutkittu,
  sekä mitä tuloksia on saatu. 
  Tiivistelmässä on lyhyt selvitys (noin 100 sanaa)
  kirjoituksen tärkeimmästä sisällöstä: mitä ja miten on tutkittu,
  sekä mitä tuloksia on saatu. 
  Tiivistelmässä on lyhyt selvitys (noin 100 sanaa)
  kirjoituksen tärkeimmästä sisällöstä: mitä ja miten on tutkittu,
  sekä mitä tuloksia on saatu. 
\end{abstractpage}


%% Preface
%%
%% Esipuhe 
\mysection{Preface}
%\mysection{Esipuhe}
I want to thank Professor Pirjo Professori
and my instructor Olli Ohjaaja for their 
good and poor guidance.\\

\vspace{5cm}
Otaniemi, 16.1.2015

\vspace{5mm}
{\hfill Eddie E.\ A.\ Engineer \hspace{1cm}}

%% Force new page after preface
%%
%% Pakotetaan varmuuden vuoksi esipuheen jälkeinen osa
%% alkamaan uudelta sivulta
\newpage


%% Table of contents. 
\thesistableofcontents


%% Symbols and abbreviations
\mysection{Symbols and abbreviations}

\subsection*{Symbols}

\begin{tabular}{ll}
$\mathbf{B}$  & magnetic flux density  \\
$c$              & speed of light in vacuum $\approx 3\times10^8$ [m/s]\\
$\omega_{\mathrm{D}}$    & Debye frequency \\
$\omega_{\mathrm{latt}}$ & average phonon frequency of lattice \\
$\uparrow$       & electron spin direction up\\
$\downarrow$     & electron spin direction down
\end{tabular}

\subsection*{Operators}

\begin{tabular}{ll}
$\nabla \times \mathbf{A}$              & curl of vectorin $\mathbf{A}$\\
$\displaystyle\frac{\mbox{d}}{\mbox{d} t}$ & derivative with respect to 
variable $t$\\[3mm]
$\displaystyle\frac{\partial}{\partial t}$  & partial derivative with respect 
to variable $t$ \\[3mm]
$\sum_i $                       & sum over index $i$\\
$\mathbf{A} \cdot \mathbf{B}$    & dot product of vectors $\mathbf{A}$ and 
$\mathbf{B}$
\end{tabular}

\subsection*{Abbreviations}

\begin{tabular}{ll}
AC         & alternating current \\
C          & celsius \\
DC         & direct current \\
IC         & integrated circuit
\end{tabular}


%% Tweaks the page numbering to meet the requirement of the thesis format:
%% Begin the pagenumbering in Arabian numerals (and leave the first page
%% of the text body empty, see \thispagestyle{empty} below).
%% Additionally, force the actual text to begin on a new page with the 
%% \clearpage command.
%% \clearpage is similar to \newpage, but it also flushes the floats (figures
%% and tables).
%% There is no need to change these
%%
\cleardoublepage
\storeinipagenumber
\pagenumbering{arabic}
\setcounter{page}{1}


%% Text body begins. Note that since the text body
%% is mostly in Finnish the majority of comments are
%% also in Finnish after this point. There is no point in explaining
%% Finnish-language specific thesis conventions in English. Someday 
%% this text will possibly be translated to English.
%%
%\section{Introduction}
\section{Introduction}

%% 
%% Leave first page empty
\thispagestyle{empty}

As technology advances, it becomes possible to build small, light-weight and yet powerful sensor platforms which can communicate wirelessly within their environment. New kind of applications are being created using the possibilities given by these sensor platforms. A common trait with all of these devices is that they need power to function, even if the power needed is minuscule. 

Traditionally wireless devices have been powered by batteries, but as the number of sensors increases, the cost of changing or charging the batteries becomes a significant part of the cost of such system. This is especially relevant for the devices which are in hard to reach areas, such as internal parts of heavy machinery, walls of bridges, high rise buildings, remote environmental sensors et cetera. In some cases the life of the battery can become a limiting factor for the lifetime of entire sensor, if the cost of installing new sensor is similar to cost of replacing the battery.

A new approach to powering devices is to harvest the energy from their surroundings using ambient energy as the power source. Examples of energy sources are solar, wind, temperature differentials and vibration. The technology to utilise wind and solar is already widely deployed and even used in large-scale power production. On a smaller scale the demand for reliable and efficient solutions has been growing steadily with the advent of low-power wireless devices. A lot of research has focused on creating suitable technologies and devices for low-power energy harvesting. 

This work focuses on powering one of such devices, namely a sensor inside a car tyre. The car tyre provides some unique challenges and opportunities, as there is a lot of energy available, but on the other hand operating conditions can be extremely harsh with large temperature ranges, extreme vibration and shocks especially in rougher road conditions.

Car tyre sensing itself has been in focus of a lot development lately, as legislation in the United States demand new cars being fitted with a pressure sensor to warn drivers about low pressure causing higher fuel consumption, wear on tyre and even elevated risk of accidents. The European Union also has laws which require Tyre Pressure Monitoring Sensors (TPMS) on new passenger cars. 

This paper provides a cursory view into current energy harvesting technologies and operational environment. Next section presents background of the field, Section \ref{sect:design} presents a design process for an electromagnetic and a piezoelectic energy harvesting system for car tyres. The results of the systems are presented in Section \ref{sect:results}, where both methods were found to produce meaningful power levels. Conclusions of the work can be found in Section \ref{sect:conclusions}. All the original material created for this Thesis can be found at https://github.com/ojousima/thesis. 


%\section{Background}
%% In a thesis, every section starts a new page, hence \clearpage
\clearpage
\subsection{Energy harvesting}
\subsubsection{Overview of methods}
First step of designing a system for energy harvesting was to identify the currently known methods and their properties. Kubba et al. \cite{Kubba2014} have done a study on tyre pressure sensor technology, they present electromagnetic, electrostatic, piezoelectric and thermal solutions as possible candidates for energy harvesting. In addition, triboelectric and magnetostrictive methods have been proposed by Bowen et al \cite{Bowen2014}. Outside of the context of tyres, Paradiso et al. \cite{Paradiso2005} present solar and radiowave harvesting techniques. Radioactive power source has been suggested by Lal et al \cite{Lal2004}. 

Electromagnetic power sources are based on Faraday's law of electromagnetic induction. A magnet and a coil are put in motion relative to each other, and the changing magnetic flux through the coils of the generator produces voltage. Current through such device is determined by load resistance. Technology is widely used in power generation, where a primary power source such as wind or flow of water provides rotation for the generator. While conventional designs use rotational movement, linear generator designs exist. Boldea and Nasar \cite{Boldea1999} provide an overview of linear generator and actuator theory. 

Electrostatic devices charge plates of a capacitor and use mechanical vibration to vary the structure of the capacitor. As the capacitance value changes with the structure, energy can be harvested from increased potential energy in capacitor. Drawback of this method is the required control electronics and high polarisation voltages needed for maximal efficiency. There are also electrostatic methods which use electrets. These electrets hold constant charge and polarisation for years and they can be used in electrostatic harvesters which do not require an external excitation source \cite{Boisseau2012}. As electret elements and electrostatic generators are not readily available, they have been excluded from this study.

Piezoelectric materials generate charge in response of mechanical stress. This stress can be caused by firmly attaching the piezoelectric element to a surface which deforms (simply supported) or by leaving one end of the element free-hanging while other end is fixed (cantilevered). Dynamics of the generator are very different for the different configurations, Kim et al. \cite{Kim2014a} provides a model for impact-based piezoelectric harvester while Erturk et al. have done in-depth analysis of cantilevered piezoelectric modelling \cite{Erturk2009}. 

Thermal solutions can be further divided into subcategories. Seebeck-effect where a temperature gradient in a semiconductor material causes voltage between poles of the material is widely used in temperature sensing, but to generate appreciable amounts of power large temperature gradients of over hundred \degree C are required according to study by Amatya et al. \cite{Amatya2010}. Such temperature gradients are not practical inside the tyre. Pyroelectric materials do not require differential of temperature, they generate energy when the temperature of the entire element changes \cite{Zhang2011}. As the temperature inside tyre remains rather constant over long periods of time, these methods are not practical for this application.

Triboelectricity generates power using friction between two materials, a classic example of this is Benjamin Franklin's experiments on charging various rods by rubbing them against different materials. A flexible triboelectric generator has been presented by Fan et al. \cite{Fan2012}. Triboelectric sheets are not readily available and their construction is complex, so triboelectric generation is excluded from this work. 

Magnetostrictive materials change their magnetic field in response to external mechanical stress. This change can be utilised to create a magnetic flux through coils as in electromagnetic generators. A magnetostrictive generator was built by Wang et al. \cite{Wang2006}. 

Solar energy can be harvested by using sun as a energy source for a thermal energy harvesting or by utilising the photovoltaic (PV) effect to generate electricity from photons hitting PV material. PV technology is mature and widely used, and PV cells attached to rim of tyre could produce ample power during summertime. PV cells would however incur extra maintenance as the rims would have to be cleaned whenever power output falls. 

Radio wave harvesting uses antennas to collect energy from ambient radio transmissions, such as WiFi- and cellular signals. Patel et al \cite{Patel2014} have built a demonstration device which uses TV broadcasts as an energy source. The tyre material dampens any Radio frequency (RF) broadcasts, which makes RF energy harvesting poorly suited for the application.

Radioactive energy harvesting resembles battery or fuel cell. A radioactive material is deposited in generator near piezoelectric cantilever. Radioactive decay charges proof mass of piezoelectric cantilever until the proof mass contacts the radioactive material by electrostatic attraction, at which point the electrical charge is balanced and piezoelectric beam begins resonant vibration as in normal piezoelectric harvesting. Such a battery has lifetime limited only by half-life of the used material. Lal and Blanchard \cite{Lal2004} present such a battery. This kind of battery would be redundant for the application, as there already exists energy in rotation of tyre which can be used to energise the cantilever. 

In conclusion, a wide range of energy harvesting technologies have been identified. As their primary properties are known, we can narrow down the suitable technologies to electromagnetic and piezoelectric. These technologies are studied further to identify optimal choice for the application.

\subsubsection{Resonance-based piezoelectric harvesting}
Piezoelectric materials produce voltage in response to mechanical stress. The effect is bidirectional, piezoelectric element can also produce mechanical strain in response to applied voltage. The material has crystalline structure with electrical dipoles in balanced state when no stress is applied. Mechanical stress unbalances these dipoles, creating element which electronically resembles a charged capacitor. 

A common approach to piezoelectric harvesting is to configure the element as a cantilever and tune the resonant frequency of the system to dominant frequency of the surrounding environment. This kind of system is shown in figure \ref{fiq:resonant_piezo}. In some applications, such as in machines running at the frequency of power grid (50 Hz or 60 Hz) this kind of frequency-tuning is relatively straightforward.

\begin{figure}[htb]
  \begin{center}
  \includegraphics[height=4cm]{images/cited/arroyo2012}
  \end{center}
  \caption{Piezoelectric generator configured as cantilever by Arroyo et al \cite{Arroyo2012}.}
  \label{fiq:resonant_piezo}
\end{figure}

This kind of resonant harvesting is challenging in tyre. The energy harvester has a very sharp peak efficiency frequencies, and dominant frequency of tyre varies with the speed of car. On the other hand, there is almost guaranteed broadband energy available from moments where tyre contacts road. There is also some research on tuning the resonant frequency of cantilevered piezoelectric harvester by Singh et al \cite{Singh2012}. They used intelligently driven SMPS to impedance-match the load to piezoelectric element. As the electro-mechanical nature of piezo means changing load changes the mechanical properties of element, resonance frequency can track the dominant frequency of system within some limits. Figure \ref{fiq:tracking_piezo} shows the tracking behaviour Singh et al achieved, resonance can be adjusted in range of 65 - 70 Hz.

\begin{figure}[htb]
  \begin{center}
  \includegraphics[height=4cm]{images/cited/singh2012}
  \end{center}
  \caption{Frequency tuning results by Singh et al \cite{Singh2012}.}
  \label{fiq:tracking_piezo}
\end{figure}

The results of Singh et al. can be considered as the state-of-art for resonance-based piezoelectric harvesting in tyre, and their power output was around $40 \mu W$ at peak efficiency. Therefore other methods have to be explored for energy harvester design.

\subsubsection{Impact-based piezoelectric harvesting}
As the resonant harvesting is not feasible in the environment inside tyre, another method would be to use an impactor to hit a piezoelectric plate on every cycle of a tyre. These impacts would provide energy once per rotation of a tyre. This method has been tried before by Manla et al \cite{Manla2009}. Their generator produced 4 mW electrical power.

As 4mW is plenty in field of low-power electronics, this approach deserves an in-depth study. Piezoelectric elements are often electrically modelled as current source with parallel capacitor or voltage source with series capacitor, as shown in figure \ref{fig:piezo_equivalents} by Kanda et al \cite{Kanda2012}. There are also a lot more complex models which account for mechanical phenomena in piezo, as well as loading effects coupling on mechanical model. For the purposes of model identification for the piezo only simplest voltage source (a) and current source (b) models are explored.

\begin{figure}[htb]
  \begin{center}
  \includegraphics[height=6cm]{images/cited/kanda2012}
  \end{center}
  \caption{Electrical equivalent models for piezoelectric element \cite{Kanda2012}.}
  \label{fig:piezo_equivalents}
\end{figure}

The model (b) of figure \ref{fig:piezo_equivalents} shows cleary that no DC current can flow in or out of piezo element. Maximum current in any cycle of piezo is limited by the open loop voltage which is seen on the terminals of piezo and the piezo capacitance in series. 

The implication for impact-based harvesting is somewhat discouraging: total amount of power optainable is limited by the frequency of impacts. However, if there is any natural resonance frequency for the piezo generator, some of the energy in impact should be in appropriate spectrum for the generator and the generator could produce decaying amount of power in-between of impacts.

\subsubsection{Electromagnetic harvesting} \label{sect:em_harvest}
Electromagnetic harvesting is based on Faraday's law of induction: A loop of wire acquires electromotive force (EMF) in response to a changing magnetic field. More formally:

\begin{equation}
  \varepsilon = - \frac{d \Phi_ {B}}{d t} , 
\end{equation}

where $\varepsilon$ is the EMF, $\Phi_{B}$ is magnetic flux through loop area, and $t$ is time. Negative sign signifies that emf opposes the change of magnetic flux. For a tightly wound coil of wire, the equation can be stated as: 

\begin{equation} \label{eq:emf}
  \varepsilon = -N_{turns} \frac{d \Phi_{B}}{d t} , 
\end{equation}

where $N_{turns}$ is the number of turns in a coil. \cite[p.999]{universityphysics}

It's important to notice that magnetic flux through wire $ \Phi_{B} $ can change for a variety of reasons: the source of field can be in motion, strength of field can vary, the coil can be in motion, and the shape of coil can vary. In an energy harvesting application in an environment with vibrations motional energy is readily available, so we focus on energy harvesting methods which either move the source of magnetic field or the coil itself.

It can be determined from equation \eqref{eq:emf} that the energy available increases with the strength of magnetic source, number of turns in a coil and rate of change in the magnetic field. 

Magnetic source can be either a permanent magnet or an electrically induced source as in induction motors. Induction-based generators require reactive power to start up, which means that any harvester design incorporating an induction generator would need a secondary power source to start the inductive generator. Hence the focus of this thesis will be in permanent magnet designs.

In addition to voltage available from the generator, it's important to consider the source impedance. A very simple electrical equivalent model of the generator is presented in figure \ref{gen_simple}, where generator is presented as a voltage source in series with lumped inductor and resistor \cite{Jirutitijaroen2012}. 

\begin{figure}[htb]
\begin{center}
\includegraphics[height=2cm]{images/own_dwg/gen_simple}
\end{center}
\caption{A simple electromechanical generator equivalent circuit.}.
\label{gen_simple}
\end{figure}

This model is greatly simplified and it does not account for factors such as effect of electromagnetic force on mechanical structure of the generator. Even with these limitations, the model is still useful as it can be used to determine an optimal load for the generator. 

The power output can be written formally as:

\begin{equation} \label{eq:gen_simple_power}
  P_{generated}(s) = \varepsilon(s)*I_{generated}(s),
\end{equation}

where $P_{generated}(s), \varepsilon(s), I_{generated}(s)$ are complex frequency-domain power, voltage and current dependent. Voltage is determined by EMF as described above. Current can be written as: 

\begin{equation} \label{eq:gen_simple_current}
  I_{generated}(s) = \frac{\varepsilon(s)}{Z_{generator}(s)+Z_{load}(s)},
\end{equation}

where $Z_{generator}(s) $ and $ Z_{load}(s)$ are complex impedances of load and generator. This equation is valid only for linear systems, so for example rectifying and converting power with switch-mode power supply (SMPS) reduces accuracy of the equation. Substituting \eqref{eq:gen_simple_current} into \eqref{eq:gen_simple_power} we obtain:

\begin{equation}
  P_{generated}(s) = \varepsilon(s)*\frac{\varepsilon(s)}{Z_{generator}(s)+Z_{load}(s)}.
\end{equation}

Total power into load can be written as:

\begin{equation} \label{eq:generator_load_power}
  P_{load}(s) = \varepsilon(s)*\frac{Z_{load}(s)}{Z_{generator}(s)+Z_{load}(s)}*\frac{\varepsilon(s)}{Z_{generator}(s)+Z_{load}(s)}.
\end{equation}

It's easy to see from \eqref{eq:generator_load_power} that if the load impedance is infinite or zero, there is no power generated. It can be shown that maximum power is generated when load impedance is complex conjugate of generator impedance, $Z_{generator}(s) = {Z_{load}(s)}^*$. Another consideration is efficiency of the generator: the electrical efficiency is defined as ratio of power flowing into load and total power generated. Equation \eqref{eq:generator_load_power} can be used to show that when load impedance is equal to generator impedance, efficiency is $ 50 \%$. Efficiency rises with the load impedance, which is why generators are rarely run at their maximum power. In our application the harvested power is minuscule compared to power available in tyre, so it makes sense to try to match the load impedance for maximum power.

In an energy harvesting application it is important to consider the validity of established theory when generator is scaled to centimetres or even smaller dimensions. Many assumptions, such as coil being tightly wound and made of thin wire might become invalid at microscale. O'Donnel et al. \cite{ODonnell2007} have done a study on the effects of scaling dimensions downwards down to millimetre range, and they concluded that power available from generator is proportional to fourth power of generator dimension for cubical generators. Another of their primary findings was that a microfabricated generator becomes more effective than a traditional wire-wound generator when design is scaled below $2 mm$ length or in $8 mm^3$ volume. It can be concluded that in this application it is reasonable to use a wire-wound generator over microfabricated one, as the generator dimensions can be an order of magnitude larger than this crossover point. 


\subsection{Structure of a tyre}

Tyres are composed of several layers with different functions. Figure \ref{fig:tyre_structure_diagram} by Gent et
al. \cite{Gent2005} shows the layered structure. From outer tread to inner lining, the layers are: 

\begin{description}
  \item[Tread] provides traction for driving, braking and cornering. Pattern and materials on
tread is a compromise between wear resistance, traction, handling and rolling
resistance
  \item[Belts] provide mechanical strength, impact resistance and keep tyre from expanding
under centrifugal forces.
  \item[Body ply] provides strength to contain the air pressure.
  \item[Innerliner] is a compound specifically designed to improve air retention in tyre.
\end{description}
In addition there are layers designed to improve tyre reliability, such as the belt
wedge which reduces shear between belts.


\begin{figure}[h]
\begin{center}
\includegraphics[height=8cm]{images/cited/gent2005}
\end{center}
\caption{Structure of a tyre \cite{Gent2005}.}
\label{fig:tyre_structure_diagram}
\end{figure}


In endurance testing of tyres the car is driven at test track in three shifts until
desired number of course driving kilometres have been reached. In outdoor testing
each company has their own proprietary test protocol. Indoor testing has standards,
which mandates pressure, ambient temperature and speed as well as time driven.
According to Gent et al. \cite{Gent2005} this indoor testing takes 34 hours of driving at 120
km/h.

In addition to endurance testing, there is high-speed testing where tyre speed is
gradually accelerated in steps of 10 km/h at regular intervals until target speed is
reached. Energy harvester should survive these tests to be considered a viable design for road conditions.


\subsection{Environment inside tyre} \label{sect:tyre_environment}
The energy harvester will be placed inside the tyre. Previous studies by Niskanen et al \cite{Niskanen2014}. have shown that the tyre will experience acceleration in all three axes. Tangential and centripetal accelerations are dominant, they can reach amplitudes up to 150 g in test fixture. In addition a study done by Löhndorf et al. \cite{Lohndorf2007} shows shock survival of up to 4 000 - 5 000 g is required for reliability. 

Temperature inside of the tyre will reach equilibrium in ambient + 5-10 \degree C, so operation temperature should be in range of -40 to + 75 \degree C to have some safety margin on top of usual ambient conditions. 

Previous work by Niskanen et al. \cite{Niskanen2014} was used to as a basis for analysis of characteristics of acceleration inside the tyre. Raw data was used to gather minimum and maximum values of acceleration as well as frequency components inside tyre. Data was gathered at 20 km/h, 60 km/h and 80 km/h speeds. Figure \ref{80_TD} shows time domain representations of the acceleration along 3 axes as shown in figure \ref{tyre_axes}.

\begin{figure}[htb]
\begin{center}
\includegraphics[height=4cm]{images/cited/matilainen2012}
\end{center}
\caption{Axes in measurement by Matilainen et al. \cite{Matilainen2012}}
\label{tyre_axes}
\end{figure}

Frequency domain representations were calculated in Matlab. There are two main contributors to base frequencies: first is the rotational frequency of tyre itself and second is the impact when the tyre deforms as it contacts the drum.. There is clearly visible series of frequency components spaced at the rotational frequency of tyre as well as shock harmonics at upper frequencies. Figure \ref{80_FFT_zoom} shows the total frequency spectrum and the dominant frequency components.

\begin{figure}[htb]
\begin{center}
\includegraphics[height=6cm]{images/matlab_figures/80kmh_timedomain_combined}
\end{center}
\caption{Acceleration of inner lining of tyre at 80 km/h in time domain.}
\label{80_TD}
\end{figure}

\begin{figure}[htb]
\begin{center}
\includegraphics[height=4cm]{images/matlab_figures/FFT-80_combined}
\end{center}
\caption{Most of the energy is found in 10-100 Hz range.}
\label{80_FFT_zoom}
\end{figure}

It's important to notice that the sensor used was piezoelectric, which forms a highpass filter as the operation of sensor is based on charge between layers. This charge dissipates over time, so the steady-state centripetal acceleration reads as zero. Any device on the rotating tyre will experience centripetal acceleration (acceleration toward centre of rotation) at the amplitude of: 

\begin{equation}
  a_{centripetal} = \omega^2 r,
\end{equation}

where $\omega$ is the rotation speed of tyre and $r$ is the radius of rotation.


%\section{Literature review?}
%\input{Literature review?}

%\section{Design?}
%\input{Design.tex?}

%\section{Results}
%\section{Results and discussion}
Experimental results from harvesters are presented in this chapter. Harvesters are tested with various loads and frequencies, 

\subsection{Experimental results of electromagnetic harvester}
This section presents the experimental results from electromagnetic harvester on vibration shaker, both on resistive load and while supplying a harvester board. 

The harvester was built to design presented in \ref{sect:emh_design}. Figure \ref{fig:emh_final} shows the completed assembly. Magnet can be seen suspended in the middle, coil is formed on the upper half of generator. Magnetic spring is formed by magnets on top and bottom sides of the harvester. 

\begin{figure}[htb]
\begin{center}
\includegraphics[height=8cm]{images/own_pic/inductive_harvester.jpg}
\end{center}
\caption{\label{fig:emh_final} Finalized electromagnetic harvester.}
\end{figure}


\subsubsection{Test setup}
This section details the test setup on vibration exciter. The electromagnetic harvester was connected to vibration exciter Brüel & Kjær type 4905 for measuring the frequency responce and output power obtainable from the harvester. Figure \ref{fig:emh_shaker} shows the test setup. Syscomp CircuitGear CGR201 oscilloscope was used to generate test signal and take the measurements from harvester. Signal from function generator was amplified by Brüel & Kjær power amplifier type 2707.

\begin{figure}[htb]
\begin{center}
\includegraphics[height=8cm]{images/own_pic/shaker_setup/emh_shaker.png}
\end{center}
\caption{\label{fig:emh_shaker} Test setup for harvester.}
\end{figure}

Regrettably the test setup did not have feedback for position of harvester, so exact displacement or acceleration of harvester is unknown. The output signal from function generator had amplitude of 6 V peak-to-peak and the gain of power amplifier was set to 9.5.

\subsubsection{Time domain results of electromagnetic harvester}
First test on the electromagnetic harvester was to measure the timedomain waveforms on various loads and frequencies. After the open loop results were obtained the tests were run again with different resistive loads to measure the power output. Finally the power output to rectifier of harvesting circuit of was measured. This section presents the test results. 

The magnet inside harvester had a notable amount of friction which had to be overcome before any output could be obtained from harvester. It was not possible to obtain very small signals from harvester, as any input strong enough to move the magnet resulted in volt-scale output. Figure \ref{fig:inductive_65_open_dry} shows an example of waveforms obtained from harvester. 

\begin{figure}[htb]
\begin{center}
\includegraphics[height=10cm]{images/own_measurement/generator_shaker/inductive_td_open_65hz_dry.png}
\end{center}
\caption{\label{fig:inductive_65_open_dry} Open circuit responce of harvester. Red is excitation waveform, blue is open-circuit voltage from harvester.}
\end{figure}

The waveforms presented in \ref{fig:inductive_65_open_dry} have some curious features: the responce from harvester is asymmetric, there is a notable valley of no output on the rising edge of the signal while no such edge is visible on falling edge. It should be noted that these valleys do not necessarily correspond to direction of gravity: the phase of input/output signal can be inversed at any point in the signal chain as the polarity of magnet, direction of winding of coils, and connection of wires can change.

There seems to be 90 \degree phase shift between excitation and responce. This phase shift was expected, as the excitation signal drives acceleration to shaker, so speed of magnet reaches maximum at zero-crossings of excitation. This observation matches well theory presented in section \ref{sect:em_harvest}: Voltage is proportional to rate of change of magnetic field. 

Amplitude of output is 2 volts and resistance of the coil was measured to be 34 ohms at DC. Inductive component of coil impedance is negligble at the frequencies of interest, so only resistive component needs to be considered. Optimal load would then be 34 ohms. When these values are substituted in time domain into equation \ref{eq:generator_load_power} in section \ref{sect:em_harvest} we obtain

\begin{align}
  P_{load}(t)& = V(t) * \frac{ 34 \Omega }{ 34 \Omega + 34 \Omega } * \frac{ V(t) }{ 34 \Omega + 34 \Omega }
  P_{load}(t)& = \frac{V(t)^2}{136}  
\end{align}

Peak power would be $ \approx 30 mW $. Root mean square (RMS) voltage cannot be accurately calculated from given values, as the waveform is not a perfect sine or triangle wave. If the waveform is approximated as triangle wave, the RMS power would be 

\begin{align}
  P_{rms}& = k * P_{peak}
  P_{rms}& = \frac{1}{\sqrt{3}} * P_{peak}
  P_{rms}& \approx 17 mW 
\end{align}

where $k$ is a constant multiplier for RMS power for triangle waves. 
If the excitation power was increased until rotor magnet audibly contacted the endstop magnets, there was no significant change in output voltage. One possible explanation is the valley in output waveform: maybe the magnet was driven to near-contact to magnet and when the acceleration was reduced the magnet was accelerated by mainly by magnetic interaction. The end result would be that the length of the valley in output waveform would vary while the output amplitude would be limited by magnetic interaction. While further exploration of this phenomenom would be interesting, the testing would be potentially destructive and therefore the tests were left to future work.

Regrettably this harvester cannot be used with the circuit designed in section \ref{sect:electronic_design} as the output amplitude is only 2 volts at any reasonable acceleration and frequency. The circuit would require minimum of 4 volts to get out of undervoltage lockout, and this is not achievable even by connecting the bridge rectifier as voltage doubler as the energy harvesting input still has two diode drops which would keep the voltage below required treshold

Next test was done by connecting the harvester to a boost circuit based on TI BQ25504 \cite{BQ25504}. BQ25504 has a boost-mode SMPS in energy harvesting input which is able to utilise input voltages down to 80 mV after startup and it can start up at roughly 330 mV. The detailed description of the circuit is given in section \ref{sect:BQ25504_schematic}.

To measure the actual power output, a current-to-voltage converter $\mu$Current \cite{muCurrent} was connected in series to harvester output. Measurement was done at scale $1 V = 1 mA$. Waveforms are shown in figure \ref{fig:inductive_vi_65}. It should be noted that the current channel might be saturated, as $\mu$Current cannot produce output higher than 1.25 V.

\begin{figure}[htb]
\begin{center}
\includegraphics[height=10cm]{images/own_measurement/generator_shaker/inductive_td_harvesting_vi_65hz_ferro.png}
\end{center}
\caption{\label{fig:inductive_vi_65} Voltage and current waveform from harvester. Red is current, 1 V equals 1 mA. Blue is voltage from the terminals of harvester before rectification.}
\end{figure}

The waveforms are as expected, there is no current flowing while voltage is low. When the voltage rises to roughly one volt, current starts to flow charging the output capacitor. When input voltage starts to decrease, no more current flows to capacitor. Accuracy of amplitude of current measurement is questionable because of potential saturation of measuring instrument.

It is worth noting that the voltage rises to open-loop maximum amplitude of 2 V as the loading on harvester decreases as the voltage on capacitor increases. This indicates that maximum theoretical peak power output of $\approx$ 30 mW is not reached at any point. 

Power waveform of harvester is presented in figure \ref{fig:inductive_power_65}. The waveform is calculated by multiplying the voltage and current. Because current is scaled at 1 mA = 1 V, the result can be read as 1 V = 1 mW. While absolute value of power is questionable because of potentially saturated instrument, the waveform itself is correct.  

\begin{figure}[htb]
\begin{center}
\includegraphics[height=10cm]{images/own_measurement/generator_shaker/inductive_td_harvesting_power_65Hz_ferro.png}
\end{center}
\caption{\label{fig:inductive_power_65} Power waveform from harvester. Pink is power, 1 V equals 1 mW.}
\end{figure}

Graphically read average power output is 0.375 mW. One possible reason for the greatly lesser power output was the capacitors in voltage doubler structure: the voltage doubler has series capacitance of 10 $\mu$F, which has reactive impedance of

\begin{align}
  X_c& = \frac{1}{2 \pi f C}
  X_c& = \frac{1}{2 \pi 65 10\mu}
  X_c& \approx 245 \Omega
\end{align}

at 65 Hz. Total output impedance of circuit would be approximately 280 $\Omega$, which would limit the output current to approximately 7 mA. This theory was tested by simulating the equivalent model of input section of harvester circuit. Simulation model and results are shown in figure \ref{}

\begin{figure}[htb]
\begin{center}
\includegraphics[height=10cm]{images/own_dwg/simulation/voltage_doubler.jpg}
\end{center}
\caption{\label{fig:simulated_doubler} LTSpice model of energy harvester input section.}
\end{figure}

The simulated data confirms the effect of input capacitor to current output of system. Current is limited to roughly 7 mA. Simulated power output was on average 2.0 mW. If the measured current is assumed to be limited by saturation, and if we assume that simulated current of 8 mA peaks would be correct, the calculated power output from experimental result would be 

\begin{align}
  P_{true}& = P_{simulated} * \frac{I_{simulated}}{I_{real}}
  P_{true}& = 0.375 * \frac{7}{1.25}
  P_{true}& = 2.1 mW.
\end{align}

After correcting the experimental current with simulated value, a lot more reasonable value of approximately 2 mW of generated power is obtained. 

This section presented the time-domain results of the electromagnetic harvester on a shaker test platform. Approximately 30 mW peak power was obtained, RMS power of 17 mW was achieved to resistive load and power output to harvester was determined to be in range between 0.4 mW and 2 mW. Next section presents the frequency domain measurements of the generator as well as studies the effect of application of ferrofluid to harvester.

\subsubsection{Frequency domain results of electromagnetic harvester}
One of the original design goals of the harvester was to provide a wide-band energy harvester solution. This section presents the frequency domain responce of the electromagnetic harvester.

Frequency domain responce was obtained by sweeping a wide-band sine signal to power amplifier and measuring the open loop responce from harvester. The first measurement was done on a harvester without ferroluid applied, figure \ref{fig:inductive_fd_dry} displays the measurement result. Above graph is output in decibels, below is the phase shift of the responce.

\begin{figure}[htb]
\begin{center}
\includegraphics[height=10cm]{images/own_measurement/generator_shaker/inductive_fd_dry.png}
\end{center}
\caption{\label{fig:inductive_fd_dry} Frequency domain responce of electromagnetic harvester before application of ferrofluid.}
\end{figure}

There is a clear resonance peak near 70 Hz. The phase shift is almost exactly 180 \degree at the resonance, which is somewhat curious result as the time domain results and theory predicted the voltage would peak at 90 \degree phase when the acceleration is at zero and speed is at highest. 

Amplitude does not have any specific meaning outside the context of this measurement and comparing output at different frequencies. It can be seen that original design goal of wide band responce has not been achieved very well, as the amplitude responce rolls off sharply below the effective frequency and maybe 20 dB / decade on frequencies above the peak. There is another resonance peak near 900 Hz, but this frequency is far above frequencies of interest for the application. 

Ferrofluid was applied to rotor magnet in attempt to reduce the effect of friction, and bode diagram was similarily plotted to figure \ref{fig:inductive_bode_ferro}.

\begin{figure}[htb]
\begin{center}
\includegraphics[height=10cm]{images/own_measurement/generator_shaker/inductive_bode_ferro.png}
\end{center}
\caption{\label{fig:inductive_bode_ferro} Frequency domain responce shows a strong resonance peak after application of ferrofluid.}
\end{figure}

The application of ferrofluid shows a strong resonance peak near 70 Hz, phase shift behaviour is similar to non-lubricated experiment. Usually the systems which have second order dynamics - such as the mass damper spring system - exhibit resonance peaks when dampening factor is low. It is therefore obvious that application of ferrofluid has resulted in lesser frictional losses. Amplitude responce is also at higher level across all frequencies, suggesting a better overall performance. 

\subsection{Experimental results of piezoelectric harvester}

\subsubsection{Time domain results of piezoelectric harvester}
The test setup of piezoelectric harvester was similar to the test setup of electromagnetic harvester. This section presents the time-domain results of the piezoelectric harvester. Details of the test setup are given in section \ref{sect:}

\subsection{Harvestering circuit results }
Text on BQ25504 instead of LTC3331, possibly testbench results of LTC3331?

\subsection{Performance inside tyre}
Testing inside tyre, matrix of power outputs. 



%\section{Conclusions}
%\section{Conclusions}\label{sect:conclusions}
In this thesis the operation environment of tyre has been presented, and reasonable choices for energy harvesting technology have been identified. Both piezoelectric and electromagnetic methods have been experimented with. The electromagnetic harvester had a characteristic of low output voltage with low output impedance. Piezoelectric harvester had notably higher output voltage but also a lot higher output impedance. A boost circuit with MPPT was built to further experiment applicability of harvester in tyre environment. 

The idea of applying ferrofluid to a small-scale electromagnetic harvester to reduce effects of friction on magnet is novel and not presented in literature to the author's best knowledge. However, the approach has been used in larger scale energy generation and wave energy harvesting. Ferrofluid application to the electromagnetic harvester notably improved the harvester performance at the resonant frequency of the harvester. 

The piezoelectric harvester was confirmed to produce 13 $\mu$W of power inside tyre at 20 km / h driving speed under 2 kN load in a chassis dynamometer test rig. The harvesting circuit failed mechanically after approximately half an hour of driving, outlining the importance of mechanical structure of device. The results obtained while the piezoelectric harvester was driven at 30 km / h under 1 kN load suggest the output power was 50 $\mu$W. A conference paper of these early results has been written and sent for peer review, and more experiments will be performed at varied speeds and loads for better characterisation of system performance. Initial results of these further tests suggests the figure of 50 $\mu$W at 30 km / h speed being reasonable. The method of determining average power production from a harvester using change of charge in supercapacitor is original approach to the author's best knowledge, while most of the literature presents the power output as a function of voltage over resistive load or continuous power output of SMPS using harvested energy.  

Average power output from coin cell batteries over a life time of a few years is comparable to output of harvesters presented in this work and elsewhere in literature. However, coin cell batteries are more economical and therefore energy harvesting systems to replace coin cells in tyre sensors are not yet commercially feasible in the author's opinion.

As the average power consumption of modern TPMS is on order of tens of microwatts, the harvester system can be concluded to produce sufficient amount of power to supply current sensor systems. However, continuous sampling of sensors and transmission of data requires tens of milliwatts and therefore solutions presented in this thesis cannot supply power to such systems. 


%% L\"ahdeluettelo
%%
%% \phantomsection varmistaa, ett\"a hyperref-paketti latoo hypertekstilinkit
%% oikein.
%%
%% The \phantomsection command is nessesary for hyperref to jump to the 
%% correct page, in other words it puts a hyper marker on the page.
\clearpage
\phantomsection
\addcontentsline{toc}{section}{\refname}
%\addcontentsline{toc}{section}{References}
\bibliography{library}
\bibliographystyle{plain}

%% Appendices
%% Liitteet
\clearpage

\thesisappendix

\section{Esimerkki liitteest\"a\label{LiiteA}}

Liitteet eiv\"at ole opinn\"aytteen kannalta v\"altt\"am\"att\"omi\"a ja 
opinn\"aytteen tekij\"an on 
kirjoittamaan ryhtyess\"a\"an hyv\"a ajatella p\"arj\"a\"av\"ans\"a ilman liitteit\"a.
Kokemattomat kirjoittajat, jotka ovat huolissaan
tekstiosan pituudesta, paisuttavat turhan 
helposti liitteit\"a pit\"a\"akseen tekstiosan pituuden annetuissa rajoissa.
T\"all\"a tavalla ei synny hyv\"a\"a opinn\"aytett\"a.   

Liite on itsen\"ainen kokonaisuus, vaikka se t\"aydent\"a\"akin tekstiosaa.
Liite ei siten ole pelkk\"a listaus, kuva tai taulukko, vaan 
liitteess\"a selitet\"a\"an aina sis\"all\"on laatu ja tarkoitus. 

Liitteeseen voi laittaa esimerkiksi listauksia. Alla on 
listausesimerkki t\"am\"an liitteen luomisesta. 

%% Verbatim-ymp\"arist\"o ei muotoile tai tavuta teksti\"a. Fontti on monospace.
%% Verbatim-ymp\"arist\"on sis\"all\"a annettuja komentoja ei LaTeX k\"asittele. 
%% Vasta \end{verbatim}-komennon j\"alkeen jatketaan k\"asittely\"a.
\begin{verbatim}
	\clearpage
	\appendix
	\addcontentsline{toc}{section}{Liite A}
	\section*{Liite A}
	...
	\thispagestyle{empty}
	...
	teksti\"a
	...
	\clearpage
\end{verbatim}

Kaavojen numerointi muodostaa liitteiss\"a oman kokonaisuutensa:
\begin{eqnarray}
d \wedge A  &=& F, \label{liitekaava1}\\
d \wedge F  &=& 0. \label{liitekaava2}
\end{eqnarray}


\clearpage
\section{Toinen esimerkki liitteest\"a\label{LiiteB}}

%% Liitteiden kaavat, taulukot ja kuvat numeroidaan omana kokonaisuutenaan
%%
%% Equations, tables and figures have their own numbering in Appendices
%\renewcommand{\theequation}{B\arabic{equation}}
%\setcounter{equation}{0}  
%\renewcommand{\thefigure}{B\arabic{figure}}
%\setcounter{figure}{0}
%\renewcommand{\thetable}{B\arabic{table}}
%\setcounter{table}{0}

Liitteiss\"a voi my\"os olla kuvia, jotka
eiv\"at sovi leip\"atekstin joukkoon:
%% Ymp\"arist\"on figure parametrit htb pakottavat
%% kuvan t\"ah\"an, eik\"a LaTeX yrit\"a siirrell\"a niit\"a
%% hyv\"aksi katsomaansa paikkaan. 
%% Ymp\"arist\"o\"a center voi k\"aytt\"a\"a \centering-
%% komennon sijaan
%%
%% Example of a figure, note the use of htb parameters which force
%% the figure to be inserted here
\begin{figure}[htb]
\begin{center}
\includegraphics[height=8cm]{kuva2}
\end{center}
\caption{Kuvateksti, jossa on liitteen numerointi}
\label{liitekuva}
\end{figure}
%%
Liitteiden taulukoiden numerointi on kuvien ja kaavojen kaltainen:
\begin{table}[htb]
\caption{Taulukon kuvateksti.}
\label{liitetaulukko}
\begin{center}
\fbox{
\begin{tabular}{lp{0.5\linewidth}}
9.00--9.55  & K\"aytett\"avyystestauksen tiedotustilaisuus (osanottajat
ovat saaneet s\"ahk\"opostitse valmistautumisteht\"av\"at, joten tiedotustilaisuus
voidaan pit\"a\"a lyhyen\"a).\\
9.55--10.00 & Testausalueelle siirtyminen
\end{tabular}}
\end{center}
\end{table}
Kaavojen numerointi muodostaa liitteiss\"a oman kokonaisuutensa:
\begin{eqnarray}
T_{ik} &=& -p g_{ik} + w u_i u_k + \tau_{ik},  \label{liitekaava3} \\
n_i    &=& n u_i + v_i.                      \label{liitekaava4}
\end{eqnarray}

\end{document}
