\section{Conclusions}
In this paper the operation environment of tyre has been presented, and reasonable choices for energy harvesting technology have been identified. Both Piezoelectric and electromagnetic methods have been experimented with. Electromagnetic harvester had a characteristic of low output voltage with low output impedance. Piezoelectric harvester had notably higher output voltage but also a lot higher output impedance. A boost circuit with MPPT was built to further experiment applicability of harvester in tyre environment. 

The idea of applying ferrofluid to small-scale electromagnetic harvester to reduce effects of friction on magnet is novel and not presented in literature to the authors best knowledge. However, the approach has been used in larger scale energy generation and wave energy harvesting. Ferrofluid application to electromagnetic harvester notably improved the harvester performance at the resonant frequency of harvester. 

The piezoelectric harvester was confirmed to produce $13 \mu W$ of power inside tyre at 20 km/h driving speed under 2 kN load. The harvesting circuit failed mechanically after approximately half an hour of driving, outlining the importance of mechanical structure of device. The results obtained while the piezoelectric harvester was driven at 30 km / h under 1 kN load suggest the output power was $50 \mu W$. A conference paper of these early results has been written and sent for peer review, more experiments will be performed at varied speeds and loads for better characterization of system performance. Initial results of these further tests suggests the figure of 50 $\mu W$ at 30 km / h speed being reasonable. The method of determining average power production from harvester using change of charge in supercapacitor is original approach to the author's best knowledge, most of the literature presents the power output as a function of voltage over resistive load or continuous power output of SMPS using harvested energy.  

Average power output from coin cell batteries over life time of few years is comparable to output of harvesters presented in this work and elsewhere in literature. However, coin cell batteries are more economical and therefore energy harvesting systems to replace coin cells in tyre sensors are not yet commercially feasible in the author's opinion.

As the average power consumption of modern TPMS in order of tens of microwatts, the harvester system can be concluded to produce sufficient amount of power to supply current sensor systems. However, continuous sampling of sensors and transmission of data requires tens of milliwatts and therefore solutions presented in this thesis cannot supply power to such systems. 