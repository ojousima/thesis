\section{Conclusions}
In this paper the operation environment of tyre has been presented, and reasonable choices for energy harvesting technology have been identified. Both Piezoelectric and electromagnetic methods have been experimented with. Electromagnetic harvester had a characteristic of low output voltage with low output impedance, and piezoelectric harvester had notably higher output voltage but also a lot higher output impedance. A boost circuit with MPPT was built to further experiment applicability of harvester in tyre envrionment. 

The idea of applying ferrofluid to small-scale electromagnetic harvester to reduce effects of friction on magnet is novel and not presented in literature to the authors best knowledge. However, the approach has been used in larger scale energy generation and wave energy harvesting. 

The piezoelectric harvester produced approximately 50 microwatts of power inside tyre at 30 km/h driving speed. The harvesting circuit failed mechanically after approximately half an hour of driving, outlining the importance of mechanical structure of device. A conference paper of these early results has been written and sent for peer review, more experiments will be performed at various speeds and loads for better characterization of system performance. The method of determining average power production from harvester using change of charge in supercapacitor is likewise original approach, most of the literature presents the power output as a function of voltage over resistive load.  

As the power consumption of modern TPMS in order of tens of microwatts, the harvester system can be concluded to produce sufficient amount of power to supply current sensor systems. However, continuos sampling of sensors and transmission of data requires tens of milliwatts and therefore solutions presented in this thesis cannot supply power to such systems. 
