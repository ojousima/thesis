\section{Introduction}

%% 
%% Leave first page empty
\thispagestyle{empty}

As technology advances, it has become possible to build small, light-weight and yet powerful sensor platforms which can communicate wirelessly with their environment. New kind of applications are being created using the possibilities given by these sensor platforms. A common feature with all of these devices is that they need power to function, even if the power needed is very small. 

Traditionally wireless devices have been powered by batteries, but as the number of sensors increases, the cost of changing or charging batteries becomes significant part of cost of any system. This is especially relevant for the devices which are in hard-to-reach areas, such as inner parts of heavy machinery, walls of bridges and high rise buildings, remote environmental sensors and so on. On some applications the life of the battery can become a limiting factor for the lifetime of entire system, as it's cheaper to replace the entire sensor with newer model than to keep using the old sensor with a new battery. 

A new approach to powering the device is to harvest the energy from it's surroundings using ambient energy as the power source. Examples of such energy sources are solar, wind, temperature differentials and vibration as the power sources. The technology to utilise some of these power sources, such as wind and solar is already widely deployed and even used in large-scale power production. On smaller scale the demand for reliable and efficient solutions has been growing strongly with the advent of wireless low-power devices and a lot of research has focused on creating suitable technologies and devices for low-power energy harvesting. 

This work focuses on powering one of such devices, namely a sensor inside a car tyre. The car tyre provides some unique challenges and opportunities, as there is a lot of energy available, but on the other hand operating conditions can be extremely harsh with large temperature ranges and extreme vibration and shocks especially in rougher road conditions.

Car tyre sensing itself has been in focus of a lot development lately, as legislation in United States demands new tyres being fitted with a pressure sensor to warn driver about low pressure and related higher fuel consumption, wear on tyre and even elevated risk of accidents. European Union is also following the trend and is drafting legislation to require Tire Pressure Monitoring Sensors (TMPS) on new cars. 

Energy harvesting system is designed to provide enough power for the sensing of the critical parameters in tire, such as pressure, temperature and acceleration. This work provides a proof of concept of an energy harvester which is capable of supporting a radio link, microcontroller and a sensor. 

Second section of this work details the background and previous work done on sensors of a tire. Third section presents the theoretical work and design of energy harvester as well as theoretical estimates for efficiency of system. Fourth section presents the experimental results from building the system. Conclusions are presented in the fifth section.

