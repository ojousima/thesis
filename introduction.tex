\section{Introduction}

%% 
%% Leave first page empty
\thispagestyle{empty}

As technology advances, it becomes possible to build small, light-weight and yet powerful sensor platforms which can communicate wirelessly within their environment. New kind of applications are being created using the possibilities given by these sensor platforms. A common trait with all of these devices is that they need power to function, even if the power needed is minuscule. 

Traditionally wireless devices have been powered by batteries, but as the number of sensors increases, the cost of changing or charging the batteries becomes a significant part of the cost of such system. This is especially relevant for the devices which are in hard to reach areas, such as internal parts of heavy machinery, walls of bridges, high rise buildings, remote environmental sensors et cetera. In some cases the life of the battery can become a limiting factor for the lifetime of entire sensor, if the cost of installing new sensor is similar to cost of replacing the battery.

A new approach to powering devices is to harvest the energy from their surroundings using ambient energy as the power source. Examples of energy sources are solar, wind, temperature differentials and vibration. The technology to utilise wind and solar is already widely deployed and even used in large-scale power production. On a smaller scale the demand for reliable and efficient solutions has been growing steadily with the advent of low-power wireless devices. A lot of research has focused on creating suitable technologies and devices for low-power energy harvesting. 

This work focuses on powering one of such devices, namely a sensor inside a car tyre. The car tyre provides some unique challenges and opportunities, as there is a lot of energy available, but on the other hand operating conditions can be extremely harsh with large temperature ranges, extreme vibration and shocks especially in rougher road conditions.

Car tyre sensing itself has been in focus of a lot development lately, as legislation in the United States demand new cars being fitted with a pressure sensor to warn drivers about low pressure causing higher fuel consumption, wear on tyre and even elevated risk of accidents. The European Union also has laws which require Tyre Pressure Monitoring Sensors (TPMS) on new passenger cars. 

This thesis provides a cursory view into current energy harvesting technologies and operational environment. Next section presents background of the field, Section \ref{sect:design} presents a design process for an electromagnetic and a piezoelectric energy harvesting system for car tyres. The results of the systems are presented in Section \ref{sect:results}, where both methods were found to produce meaningful power levels. Conclusions of the work can be found in Section \ref{sect:conclusions}. All the original material created for this Thesis can be found at https://github.com/ojousima/thesis. 
