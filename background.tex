\section{Background}
%\section{Aikaisempi tutkimus}

\subsection{Environment inside tire}
The energy harvester will be placed inside the lining of the tire. Previous studies in department have shown
that the tire will experience acceleration in all three axes. Tangential and centripetal accelerations are dominant,
they can reach amplitudes of up to 150g in text fixture. 

In addition, literature suggests shock survival of up to n hundered g is required for reliability. 

Temperature inside of the tire will reach quilliberium in ambient + 5-10 \degree C, so operation temperature should be
in range of -40 to + 75 \degree C to have some safety margin on top of usual Finnish ambient conditions.

As cars move from place to place, any energy source from environment outside the tire cannot be relied upon. 

\subsection{Power requirements of a harvester}
The sensor system will be in three distinct states. One is sleeping, conserving power as much as possible while car is not moving.
Second state is measuring, when the radio connection is off but electronics are active and gathering data.
Third state is transmitting, when the data is relayed to drive computer in car.

The power requirement of system create additional constraints on energy harvesting method, as the energy harvester must be able to supply 
enough energy and power to system. 

Energy and power consumption are estimated by reviewing a few suitable components and their power requirements. 
Energy management is handled by a specialized integrated circuit (IC), for example LTC3331 \cite[p. 2] {Technology}.



%%
%% Three levels of hierarchy in sectioning should be enough

